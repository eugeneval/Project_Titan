\documentclass[11pt]{article}
\usepackage{textcomp}
\author{Eugene Valetsky}
\title{Project Titan} % TODO: real title

%%%%%%%%%%%%%%%%%%%%%%%%%%%%%%%%%%%%%%%%%%%%%%%%%%%%%%%%%%%%%%%%%%%%%%%%%%%%%%%
% DOCUMENT
%%%%%%%%%%%%%%%%%%%%%%%%%%%%%%%%%%%%%%%%%%%%%%%%%%%%%%%%%%%%%%%%%%%%%%%%%%%%%%%
\begin{document}
\maketitle
\tableofcontents

\section{Introduction}
The Unmanned Aerial System (UAS) Challenge was launched by the Institution of Mechanical Engineers (IMechE) in 2014 `with the key objectives of developing professional engineers and inspiring the next generation'\cite{IMechE_about_uas}. The main goal is to design and build a UAS in order to autonomously deliver humanitarian aid such as medical supplies in a disaster zone. This requires a broad range of skills, from project management to airframe design to avionics system programming. To add to the challenge there is a strict weight limit and stringent safety regulations. Working together with another student, Ismail Ahmad, our goal is to follow the design and build cycle all the way through to competing at the IMechE UAS competition in July 2018.

One of the greatest individual challenges as part of the project is creating an autonomous flight system. The UAS must be capable of quickly and safely navigating to and locating a target using computer vision, dropping a payload accurately, and returning to base. This requires an understanding of the UAS’s flight dynamics as well as good programming and wiring capabilities. This is the aspect that I will be focusing on for my individual project, although I will be assisting with many parts of the project.

\section{Design}
\subsection{Mechanical Design}
The concept for our UAS uses a hybrid symbioisis between a \emph{Micro Jet Engine}, henceforth refered to as MJE, and an external multi-rotor. The MJE is gimballed to always remain vertical and only provides lift, while the multirotor rotates around it providing stabilisation and control. This configuration means that standard quadcopter flight control software can be used rather than needing to come up with custom architecture.

The MJE provides a higher thrust to weight ratio than the equivalent electric motors and batteries (see \ref{thrust_to_weight}). It does this without introducing the vibration issues of a piston engine, which would seriously impact stability and control on such a lightweight design. With the current MJE and electric motors we hope to lift a payload of 3kg while remaining under the 6.9kg weight limit.
% TODO: 3 or 4 kg?
\subsection{Control Design}
Flight control will be handled with a Pixhawk running the PX4 flight stack. This will be coupled to a companion computer running software based on the DroneKit SDK, communicating with the PixHawk using the MAVLink protocol. The companio

\section{Simulation}
\subsection{Concept}
One of the main concepts behind the UAS concept is the idea that, if the MJE is gimballed to remain vertical, and its thrust vector is through the center of gravity of the vehicle, it exerts no horizontal forces or moments on the rest of the vehicle. This means its effect on the stability and control of the vehicle can be ignored. In turn, this means that regular quadcopter flight control software can be used, such as the PX4 flight stack.

Being able to use PX4 firmware running on a Pixhawk is crucial. The UAS will come to cost approximately \pounds2,500. Using home-made flight control software is therefore extremely risky. PX4, on the other hand, is a project that has been worked on by thousands of people for years, and is infinitely more reliable than anything we could put together from scratch.

Therefore, it was decided to build a simulation to test the validity of the concept that the MJE can be ignored.

\subsection{Equations of Motion}
The vehicle is split into two sections, the outer quadcopter frame, \emph{Quad}, and the gimballed section containing the MJE, \emph{Jet}. A cartesian reference frame was chosen, with the xy plane being the horizontal plane and z being height. +x is forward, +y is right, +z is up. A rotation about the x axis is roll, about the y axis is pitch, and about the z axis is yaw.

The gimbal is controlled by two servos, one moving the gimbal in roll and one in pitch. (There is no need for yaw control of the jet.)
\subsubsection{Inertia}
The quad is modeled as an inner thin, hollow cylinder with four cyclindrical rods extending outwards as the quad arms. The motors are point masses on the ends of the frames. This results in the following inertias about the three defined axes:
\begin{eqnarray}
    % TODO: seperate symbols for quad and jet inertia
    I_x = I_y & = & \underbrace{\frac{\pi \rho h}{12}(3(r_2^4-r_1^4) + h^2(r_2^2-r_1^2))}_{Frame} \nonumber \\ & + & 4 \cdot \underbrace{\sin 45 \cdot (\frac{M_R L^2}{12}+M_R(r_2+\frac{L}{2})^2)}_{Rods} \nonumber \\ & + & 4 \cdot \underbrace{\sin 45 \cdot M_M(r_2+L)^2}_{Motors} \\
    I_z & = & \underbrace{\frac{\pi \rho h}{2}(r_2^4-r_1^4)}_{Frame} \nonumber \\ & + & 4 \cdot \underbrace{\frac{M_RL^2}{12} + M_R(r_2+\frac{L}{2})^2}_{Rods} \nonumber \\ & + & 4 \cdot \underbrace{M_M(r_2+L)^2}_{Motors}
\end{eqnarray}

The jet is modelled as a cylinder. However, it does not rotate about its origin in x and y, but about the appropriate servo. Including the inertia of the servos themselves (see Appendix \ref{servo_info}) results in the following inertias:
\begin{eqnarray}
    I_x = I_y & = & 3.89\times10^{-4} + M_J*3r_J^2 + h_J^2 + M_Jl_c^2 \\
    I_z & = & M_J*r_J^2
\end{eqnarray}

\subsubsection{Forces}
Our gimbal design will transmit forces, but not moments. Thus, from a forces perspective, the vehicle is treated as a single unit. The total mass is
\begin{equation}
    M_{TOT} = \underbrace{\rho\pi h (r_2-r_1)}_{Frame} + 4*\underbrace{\rho\pi L r_r}_{Rods} + 4*\underbrace{M_M}_{Motors} + 2*\underbrace{M_G}_{Servos} + \underbrace{M_J}_{Jet}
\end{equation}



%%%%%%%%%%%%%%%%%%%%%%%%%%%%%%%%%%%%%%%%%%%%%%%%%%%%%%%%%%%%%%%%%%%%%%%%%%%%%%%
% APPENDIX
%%%%%%%%%%%%%%%%%%%%%%%%%%%%%%%%%%%%%%%%%%%%%%%%%%%%%%%%%%%%%%%%%%%%%%%%%%%%%%%
\appendix
\section{Servo Information} \label{servo_info}
Servo information was based on a sample servo that may well end up being used on the vehicle, the Futaba BLS177SV.

\begin{center}
\begin{tabular}{ccc}
    Torque (at 6.6V): & 34 kg-cm \\
    Speed (at 6.6V): & 0.12sec/60\textdegree{} \\
    Weight: & 79g \\
\end{tabular}
\end{center}

We can see from the datasheet that it takes the servo 0.12 seconds to rotate 60\textdegree{}. Assuming it takes 0.01 of those seconds to accelerate to full speed, this gives an acceleration of $873rad/s^2$. Since we know it exerts a torque of 0.34kgm, this means its inertia must be $3.89\times10^{-4}$.
% TODO: perhaps follow this up experimentally?

\section{Thrust to Weight Ratio Calculations} \label{thrust_to_weight}
% TODO: add stuff here



%%%%%%%%%%%%%%%%%%%%%%%%%%%%%%%%%%%%%%%%%%%%%%%%%%%%%%%%%%%%%%%%%%%%%%%%%%%%%%%
% BIBLIOGRAPHY
%%%%%%%%%%%%%%%%%%%%%%%%%%%%%%%%%%%%%%%%%%%%%%%%%%%%%%%%%%%%%%%%%%%%%%%%%%%%%%%
\begin{thebibliography}{11}

\bibitem{IMechE_about_uas}
  https://www.imeche.org/events/challenges/uas-challenge/about-uas-challenge

\end{thebibliography}

\end{document}
